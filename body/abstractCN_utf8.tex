\cabstract{

随着互联网的迅猛发展,极大地推动了人们对无线互联网及移动互联网应用的需求。
宽带网络以及无线网络的日新月异的技术发展,也使得人们感受到了宽带无线网络对人们生活的影响。
同时,无线频谱资源的稀缺性与多媒体业务日益增多的需求之间的矛盾,
对无线资源管理的研究提出了新的挑战。

本文从理论与应用相结合的角度出发,选择无线网络中数据链路层层为研究背景,
特别是对当前数据链路层资源管理相关的三个技术关键点,进行了深入细致地研究。
其中包括高速移动台切换协议流程及信令保护,呼叫接纳控制以及相应的资源分配和调度机制,
以及应用博弈论方法解决多媒体业务资源分配和调度的研究工作。
主要的研究工作包括以四个方面。
\begin{enumerate}[(1.)]
\item 
针对高速移动台切换问题,建立了以移动速度为基础的切换成功概率模型。
通过该模型来考察速度对切换协议信令影响的关系。
提出一个自适应的FEC保护方案来匹配不同移动速度,进而改善切换成功率。
仿真实验证明该方法简单有效。

\item 
根据多媒体业务的不同,
建立一个针对网络底层资源分配参数与网络高层的业务服务质量的映射模型,
实现不同业务用户的QoS测度以及系统整体性能的有效测量。
并且,基于这个映射模型,提出了一个基于系统效用和用户自身效用兼顾呼叫接纳控制与资源分配的算法,
实现不同业务Qos以及系统资源的在线管理。
该算法以用户服务质量效用为最大化目标,在高负荷下依据用户业务负载情况调整接纳与分配策略。

\item 
针对用户资源竞争与分配的问题,提出了一个新的资源分配议价博弈模型及相应的分配算法。
从理论上分析了所提出的议价模型合理性,以及其可以满足纳什议价所提出的公理约束。
通过对议价能力的分析与讨论,提出了基于用户应用特征参数值的用户议价能力的函数定义形式。
仿真结果表明,所提出的议价博弈模型可以有效地描述用户之间的资源争用问题。
所提出的资源分配算法在保证系统资源利用率的同时,又可兼顾各个用户之间的自身利益。

\item 
针对网络中业务逐渐增多的趋势,提出通过概率连续随机变量来描述用户的业务类型。
与传统的离散分类不同,新的方法可以连续细致地刻化用户业务特征。
并且根据网络的实际情况,提出在不完备信息的情况下通过构造Bayesian博弈模型的方法来寻求使用户满意的资源分配策略。
通过理论分析与仿真实验证明,所提出的业务描述方法及博弈算法,可以有效地解决在多用户竞争下的资源分配问题。
\end{enumerate}
}
\endinput

