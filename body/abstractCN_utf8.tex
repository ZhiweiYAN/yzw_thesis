\cabstract{
信息技术与互联网的迅猛发展,极大地推动了人们对无线互联网及移动互联网应用的需求。
宽带网络以及无线网络日新月异的技术发展,也使得人们感受到了宽带无线网络对人们生活的影响。
然而,无线频谱资源的稀缺性与多媒体业务日益增多的需求之间的矛盾,
对无线资源管理的研究提出了更多的挑战。

本文从理论与应用相结合的角度出发,以无线网络的数据链路层为研究背景,
对数据链路层的资源管理与控制中的关键技术进行了深入细致地研究,取得了以下主要的研究成果。

\begin{enumerate}[(1)]
\item 
针对高速移动终端的基站切换问题,通过分析移动终端的移动速度对误比特率的影响,
以及移动速度与相邻基站交叠区域距离的关系,
建立了以基于移动速度的基站切换成功概率模型。
此概率模型有效地描述了移动速度在频谱偏移和时间延迟上对切换协议信令接收与发送的影响,并直接建立了切换成功概率与移动速度之间的关系。
然后,基于此模型,提出了一个移动速度自适应的FEC保护方案。仿真结果表明,基站的切换成功率与切换效率可以有效地得到改善和提高。

\item 
通过分析多媒体中的各种不同业务的特征,
建立了一个针对网络底层资源分配参数与网络高层的业务服务质量(QoS)之间关系的映射模型,
实现不同业务用户的QoS水平统一标定以及系统整体性能的有效测量。
并且,基于这个映射模型,提出了一个基于系统效用和用户自身效用兼顾的呼叫接纳控制与资源分配的算法,实现不同业务QoS要求以及系统资源的在线管理。
该算法以用户服务质量效用为最大化目标,在高负荷下依据用户业务负载情况调整接纳与资源分配策略,可以有效地提高资源利用及用户的QoS水平。

\item 
针对用户资源竞争与分配的问题,提出了一个新的资源分配议价博弈模型及相应的分配算法。
从理论上分析了所提出的议价模型合理性,以及其可以满足纳什议价所提出的公理约束。
通过对议价能力的分析与讨论,提出了基于用户应用特征参数值的用户议价能力的函数定义形式。
实验的结果表明,所提出的议价博弈模型可以有效地描述用户之间的资源争用问题。
所提出的资源分配算法在保证系统资源利用率的同时,又可以兼顾单个用户自身通信资源的合理利用。

\item 
针对网络中业务逐渐增多的趋势,提出了通过概率连续随机变量来描述用户的业务类型的思想。
与传统的离散分类不同,新的方法可连续细致地刻化用户业务特征。
并且,根据业务随机分布情况,提出在网络不完备信息的情况下,通过构造Bayesian博弈模型的方法来寻求使用户满意的资源分配策略。
通过理论分析与仿真实验证明,所提出的业务描述方法及博弈算法,可以有效地解决在不完备信息下的多用户竞争与资源分配问题。
\end{enumerate}
}
\endinput

