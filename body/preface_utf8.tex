\chapter{绪论}
\echapter{Preface}
\section{研究背景与意义}
\esection{Research Background}
二十一世纪以来,电子通信技术领域发生了许多重大的变化。
其中包括传统的固定电话交换网络,包交换网络,高速局域网,以及无线移动通信网络。
从无线通信网络使用的技术来看,早期的模拟蜂窝网络已经完全被数字通信网络所代替。
同时,无线数据终端的类型也日趋多样,包括移动电话、普通台式计算机、笔记本电脑或平板电脑等等。
而且,它们所使用的数字通信技术也多种多样,包括GPRS,CDMA,WCMDA,OFDM等等。
与此同时,互联网(Internet)发生了翻天覆地的变化。
首先其所承载的内容也发生了重大变化。从单一的文字和低分辨率的图片发展为集交互性与实时性并重的多媒体内容。
其次从连接的需求看,从固定的连接向无线的连接迅速转变。
也正是由于互联网的发展,使得人们对于通过无线数字通信技术来连接互联网,并且传输海量的数据信息成为了现实中迫在眉睫的问题。
进而,使得无线通信技术所承载的业务也从单一的话音业务转变为宽带多媒体数据。
并且,这一趋势使得无线通信网络的发展与互联网的发展紧密地结合在一起。
\par 从技术的角度讲,如何研发并成功布署一个无线宽带网络而言,面临着诸多旧的和新的问题。
\par 为了能够满足用户对多媒体海量数据业务的需求,无线宽带网络系统必须能够承载每秒数兆比特流的数据,并能够安全快速地传送到每一个终端用户的设备上。
同时还可以支持不同用户等级的服务质量要求(Quality of Service, QoS),及不同类型数据的业务等级服务质量要求,例如,话音、数据、视频或其它类型的多媒体等等。
从现实的宽带网络技术布署情况看,目前的有线宽带技术在室内已经比较好地支持当前的多媒体业务运营。
所以,人们希望无线移动宽带网络技术也能够在空中将有线宽带网络所能支持的业务承担起来,把各种业务的数据通过空中接口发送到用户的移动电话、笔记本电脑、平板电脑或是其它的移动终端上。
用户对无线宽带技术的要求与有线宽带技术一样,希望在服务质量、可靠性与安全性上都能够得到保证。
为了能满足这些严格的要求,无线宽带网络技术的发展面临着下面的关键技术需求:
\begin{itemize}
\item   在一个不确定性的无线空间环境中,可靠的数据发送及接收方案。
\item   在有限的无线频谱资源与多样且大量宽带业务之间作出平衡。
\item   同时支持不同服务质量要求的多个业务流。
\item   支持无缝的小区或基站切换技术和漫游技术。
\item   针对电池供电的移动设备,开发相对应的低功耗技术。
\item   兼容基于IP协议的应用,以保证快速且低成本的应用布署。
\item   可靠的数据安全技术。
\end{itemize}

在处理这些技术难点时,往往会出现这种矛盾的情况:改善了一个技术指标的同时,又把另一个技术指标中问题恶化了。
比如,系统容量与覆盖范围就是这样一对矛盾。
所以,在设计无线宽带系统时,要综合考虑各种相互矛盾的因素,根据实际情况做出适当的折衷与平衡。
总的来说,从技术上可归纳为以下几个方面的挑战:无线信道、频谱稀缺、服务质量和移动性。
%%
\begin{enumerate}[1)]
    \item {无线信道:}
对于宽带无线技术而言,最基本的技术挑战是传输介质本身。
在有线通信信道中,一个物理实体的连接,例如同轴线缆或是光纤,将信号从发送端传递到接收端。
而在无线的传输环境中,信号的传递依赖于一个比有线环境复杂得多的空间电波传播机制。
对于大多数的无线应用需要在不可视距(Non Line of Sight, NLOS)环境下传递信号。
在这样的传播过程中,空间中大小不同的物体、距离上的远近、发送端与接收端之间的相互运动、以及空间电波噪声干扰等等都会影响信号的传递质量。
因此,设计一个在此复杂条件下工作性能良好的数字通信系统,同时又要满足高速率的数据传输以及发送端与接收端相互间的高速运动,
对研究者而言,是一个艰难的挑战。
具体来讲,对于宽带无线信道,存在以下技术难点:
\begin{enumerate}[(1)]
\item 不可视距下的路径传播损耗(path loss):在不可视距下,接收端的信号功率随距离衰减的速度比可视距下会更快。
接收信号的功率会受到天线的高度、地形等因素的影响。
这种影响,一般也称作损耗。它指的是无线电磁波在传输过程中由于传输介质的因素而造成的损耗。
在固定电话通信等有线通信的过程中也有路径损耗。但是相比与无线信道,它们的路径损耗相对简单。
\item 阴影效应(shadow effect):在无线通信系统中,移动台在运动的情况下,由于大型建筑物和其他物体对电波传输路径的阻挡,会在传播接收区域上形成半盲区,从而形成电磁场阴影。
这种随移动台位置的不断变化而引起的接收点电磁场强度起伏变化的现象叫做阴影效应。
阴影效应是产生慢衰落的主要原因。
慢衰落会使信号严重衰减,这也是在通信技术发展中要克服的重要问题之一。
\item 多径衰落(multipath fading): 由于通信地面站天线波束较宽,受地物、地貌等诸多因素的影响,使接收机收到由同一个信号出发,
    经折射、反射和直射等几条路径到达的多个电磁波信号,这种现象就是多径效应。
这些经不同路径到达的电磁波射线相位不一致且具有时变性,导致接收信号呈衰落状态;又由于这些电磁波射线到达的时延不同,会导致码间干扰。
若此多条射线强度较大,且时延差不能忽略,则会产生误码。
这种误码仅靠增加发射功率是不能消除的。因而,我们把多径效应产生的衰落叫多径衰落。它是无线通信中产生码间干扰的重要根源,
对于数字通信质量产生十分严重的影响。
\item 码间干扰(intersymbol interference ,ISI):在多径的环境下,无线信号经不同的路径到达接收端。
如果此时的时间延迟较大或者说这个时间延迟已经占了传输信号符号周期的一部分,那么这个传输信号也许会在下一个符号周期内被接收端收到。
这就影响了下一符号的接收。
在高速率通信中,每个传输码的时间短,一个小的时间延迟也会造成码间干扰。
对于宽带无线通信而言,它技术难度很大。
均衡化是解决这个问题的传统方法,但需要消耗更多功率。
OFDM技术近十年来的发展,对这个问题的改善起来很大的作用。
但是,另一方面,在发送机与接收机之间的相对运动会造成频率扩散(多普勒扩散)。多普勒扩散的产生又与运动的速度与载波频率相关。
所以,对于宽带系统,它会使信噪比的下降(Signal-to-noise ratio, SNR),进而增大载波恢复和同步的难度。
这也让此问题成为了一个重要的技术难点。
\item 噪声:加性高斯白噪声(Additive white Gaussian noise, AWGN) 在所有的通信信道中都存在。
在接收端的热噪声的大小与带宽是成正比的。
所以,宽带通信的噪声基底比传统的窄带通信的噪声基底高很多。
较高的噪声基底并伴随着较大的路径损失会使宽带系统的覆盖范围减小。
\item 其它干扰:由于可用的频谱资源有限,用户需要共享频谱资源。
不同用户之间的使用会造成相互的干扰。
在一个以系统容量为驱动的网络中,这种干扰有时会比噪声造成的负面影响更为严重。
\end{enumerate}
%%
\item{频谱的稀缺}

宽带无线通信所遇到的第二个大的问题是无线频谱资源的稀缺。
因为世界各国的无线电管理部门对无线电频谱的管理十分严格,且只分配给一段有限的频谱资源进行商用或民用。
所以,无线通信的宽带系统设计与研究就必须面临这一问题:高效地使用有限的无线电频谱资源来容纳数量日益增长的用户和各式各样的宽带应用。
\par 高效利用频谱的最基本的方法是在空间地理上构建蜂窝结构的网络,在技术上引入频率复用思想。
通过降低发射机发射功率来限制基站的覆盖范围,
然后再通过布署多个低功率的基站来达到较大范围的地理覆盖。
每一个基站覆盖一个小区(cell)。
每一个小区可再通过定向天线技术划分为多个扇区。
通常,一组小区或扇区构成一个簇。可利用的频谱被分配到簇内的小区或扇区中。
从而可以减少相互之间的干扰。各个簇之间可以复用相同的频率,也称为频率复用技术。
为了达到较大容量和高效的频谱利用率,频率复用必须最大化。
然而,这就会出现前面所说到的矛盾,增加了复用率会导致严重的电波干扰。
所以,设计一个可靠的传输和接收方案,支持低信噪比的传输环境,并能够抗干扰的方法是当前的研究重点。
其中,多天线技术是目前抗干扰技术中的研究热点。

如果只是从技术角度寻求频谱的利用率与系统容量的简单方法而言,减小基站覆盖半径和增加基站数目就可以满足这一要求。
但是实际应用中,对于一个商业通信系统,这种简单的方法成本过于昂贵。
所以,除了使用蜂窝结构和最大化频率复用,另外的一些技术也可用来提高频谱的利用率和系统的容量。
\begin{enumerate}[(1)]
\item 自适应调制编码(Adaptive modulation and coding,AMC):
因为不同的用户或每个数据包当前的传输环境都可能不同,所以可以相应地改变其数据的调制与编码方式。
通过选择那些高阶的调制方法或编码方法,来提高用户数据的传输效率,增大系统容量。
\item 空间复用:这个方法的思想是通过多个天线将多个不相关的数据流同时发送出去,接收机也相应地使用多个接收天线和相应的信号处理方法来还原数据。
只要不同的天线对应的信道相关性很小,在整个传输过程中,数据完整性和可靠性就可得到保证。
\item 有效的多路访问技术:除了让每一个用户尽可能充分利用频谱资源,设计有效的方法来让多个用户共享资源也是十分重要的。
这部分的工作目前主要集中在网络数据链路层。通过分析用户及用户业务类型信息,也可以使得资源的利用率提高。本论文的大部分工作会将集中在这一方面。
因此,通过充分挖掘系统本身的能力,以及对各个系统参数(如系统容量或覆盖范围)的折衷方案的使用,才能在一个可行的成本下,提供给用户满意的通信服务。
\end{enumerate}
\item{服务质量}

服务质量(QoS)从概念上泛指为多个用户提供通信服务的效果。
如果从传统的通信技术系统指标上讲是指吞吐量(throughput),数据包的差错率(packet error rate),时间延迟(delay),时间抖动(jitter)。
这些系统指标是与用户的业务是相关的。不同的业务会拥有不同的系统指标值。
例如,对于宽带无线网络而言,它会同时支持不同种类的业务类型,如话音、数据、视频和其它的多媒体数据。
每一种都有可能有不同的数据特征和相应的QoS要求。表\ref{tb:preface_qos_parameters}描述了一些常见的业务类型及相应的指标要求。

\begin{table}
\caption{典型宽带应用的各项QoS参数比较} \label{tb:preface_qos_parameters}
\begin{center}
\begin{tabular*}{0.98\textwidth}{p{1.8cm}p{2.5cm}p{2cm}p{2.5cm}p{2.5cm}p{2cm}}
\toprule
参数&交互式游戏&语音&流媒体&普通数据&视频\\
\midrule
速率要求&50Kbps-85Kbps&4Kbps-~64Kbps&5Kbps-384Kbps&10Kbps-100Mbps&$>1$Mbps\\
现实应用&交互式游戏&VoIP&在线音乐,在线视频,语音聊天&网页访问,电子邮件,即时通信,文件下载&IPTV,P2P的视频共享\\
传输特点&实时&实时连续&连续有突发&非实时,具有突发特性&连续\\
丢包要求&不允许&$<1\%$&声音$<1\%$, 视频$<2-5\%$&不允许&$<0.1\%$\\
延时抖动&{\small$<20$}毫秒&$<20$毫秒&$<2$秒&N/A&$<2$秒\\
延时要求&{\small $<50-150$}毫秒&$<100$毫秒&$<250$毫秒&$<1-10$秒&$<100$毫秒\\
\bottomrule
\end{tabular*}
\end{center}
\end{table}
另外,除了应用级的QoS要求以外,网络运营本身也会做一些策略性的QoS的控制。
例如,针对承载业务的不同,对不同的用户采取相对应的服务。
或是根据用户订购服务的不同,对用户的数据区分对待。
从QoS的角度来看,业务、服务等级以及用户类型都会对用户所体验到的QoS产生影响。
如何在网络中采取有效的措施来容纳不同层次的QoS要求,对于资源稀缺的无线网络来说是一个重要的研究课题。

除此之外,由于IP网络的优势,未来的网络一定是以IP为核心的网络。结合IP层的特点来改善QoS也是一个研究的热点。
\item{移动性}

对于终端用户来说,移动性是无线通信技术中最具有价值的服务之一。
但它也同时将一些难点问题引入进来。
其中,有两个最为重要。一是漫游技术(roaming)。通信网络要提供一种手段使得网络可以定位在网的空闲用户(inactive users),并且可以让他随时被激活;并且完成一些数据收发的初始化的的操作。
二是切换技术(handoff 或handover),提供技术保证使得在线用户(ongoing users)在终端移动的情况下仍旧保持数据或话音通信的连续性。
这两种技术都属于移动性管理的范畴。它们为更好的用户体验提供技术保证。
\begin{enumerate}[(1)]
\item 漫游技术:为了定位正在漫游的用户,通常采用的方法是使用中心数据库来存储用户的位置信息,并随时进行更新。
当用户从一个位置移动到另一个位置的时候,它会向网络中心数据库报告自己的移动情况。
另一方面,为了找到一个在网用户,网络也会在初始化一个数据或话音服务会话的时候,主动地向相关的基站发送寻呼的消息(page message)。
接收寻呼消息的基站数目依赖于用户的移动情况和消息发送的频率。
在无线资源管理中,如何对消息发送的频率与接收的范围进行合理地设置也是无线资源管理的一个重要的课题。
\item 切换技术:这个问题涉及到如何保证用户在移动的过程中正在进行的数据或话音通信不中断。
首先遇到的问题是要检测和决策进行切换的时机。
其次是分配相应的资源来保证切换的成功。
再次是设计相应的信令流程来实际完成这个切换过程。
因为切换的时机不容易预测且规律性不强,这也给切换判断过程带来了一定的困难。
在设计切换算法时,要在掉话率(dropping probability)与切换率(handoff rate)之间做折衷的考虑。
如果切换频率过高,会使得信令过多,影响通信的质量。
如果切换决策不及时,会使得通信中断,也会影响用户的体验。
在切换技术中,另一个重要的研究问题是,在切换过程中,设计一套机制保证切换所需的无线资源。
否则,通信也会由于无线资源不能及时调度到位造成用户通信中断或用户的通信质量下降。
例如,有些通信系统会预留一部分资源专门针对切换使用或是切换用户会优先使用这部分无线资源。

同时,由于无线IP(Internet protocol)互联网的发展,在IP层上的移动性管理也日益显得重要起来。
传统上,移动性管理主要是针对第二层(数据链路层)做设计的。
而在无线IP的网络中,用户的IP在通信服务的会话过程中有时要求是固定的。
例如,网页缓冲服务或数据多播服务。这也给基于IP的无线移动管理带来了新的挑战。
另外,通信标准的多样化,以及通信发展本身新旧网络的更新,使得不同结构或技术标准的网络同时存在。那么,这种网络的异构性使得用户接入方式可以有选择。例如,可以通过WLAN或WiMAX接入,也可通过GPRS或3G接入。
那么,IP移动管理技术同样要解决在异构网络中的漫游与切换问题。
\end{enumerate}
%
\item 其它
\par 在无线网络中,另外两个问题也不容忽视。
一个是功率控制与节电技术,另一个是数据加密与通信安全。
因为绝大多数的用户无线终端设备是通过电池来提供能源的,所以功率控制与数据传输控制的结合就显得十分重要。
在给定的电池消耗下,寻找高效的传输方案,或是设计简单的协议流程,以及计算量更小的信号处理算法都属于这一个范畴。

同样,数据与通信的安全在任何通信系统中都是十分重要的。
由于无线通信的信号是在空中传播的,任何人都可能侦听到其中的电波。
如何确保任何人都不能非法侵入用户的正常会话中。这在通信安全中极其重要的。
另外,从运营商的角度看,安全也可指防止非受权的用户可以接入和使用网络的资源。
这种安全措施目前在网络中的物理层、网络层以及服务应用层中都被采用和实现。
\end{enumerate}
\section{无线资源管理与数据链路层控制的研究进展}
\esection{Resource Managment and Data Link Layer Control}
上一节提到,因为无线频谱的资源是十分有限的,所以无线资源管理与分配一直是无线互联网及移动多媒体通信网络的研究热点之一。
从定义来说,无线资源管理主要是针对无线通信网络中的空中接口资源进行有效管理以及分配。
首先它的作用体现在可以给每个单独的用户提供满意的通信服务。
譬如,当网络出现用户数量变化或是用户之间负载不均衡时,可以灵活地调整资源的分配策略来保证不同用户的通信质量。
其次是,在无线信道出现大的质量波动时,通过诸如调制方式的改变来满足用户的通信需求。
再次是在满足用户通信需求的前提下,能够充分利用系统所提供的无线资源,不使其浪费。

相应地,无线资源管理的评价也分为两类具体的性能指标:一类是指从用户角度来评判,称为用户级别。
这类指标是在系统设计时往往优先考虑的,并且它与用户的类型,承载的业务本身关系很大。
另一类是从系统的角度来评价,被称为系统级别。如系统的吞吐量,频谱的利用效率,系统发射功率或效率等等。

%这里,如果我们从网络分层结构上来看待无线资源管理,可将其分为物理层资源管理技术、MAC层资源管理技术和交叉层的资源管理技术。
%
传统上来讲,资源管理包含的内容非常广泛。目前研究的重点包括以下几个方面的内容:

\begin{itemize}
\item 切换控制:当移动终端从一个基站覆盖域移动至另一个基站的覆盖域时,
    或者当网络负载控制及维护等原因使得用户被迫切从当前的服务小区或基站换到其他小区或基站时,保证该用户业务不中断。
\item 接纳控制:在保证己有用户QoS要求的同时,尽可能接纳更多用户,提高系统容量。 同时,也要降低新呼叫的阻塞率和切换呼叫的掉线率,使网络的综合性能指标最大化。 
 \item 调度机制:使各个在线用户合理地使用系统的可用资源,
    为各个用户分配数据速率和分组长度。
\item 负载控制:当网络出现过载或临近过载的情况时,及时进行调整,使系统迅速可控地恢复到正常状态,以保证网络的稳态运行。
\item 功率控制:主要目的是在维持链路通信质量的前提下,尽可能减小功率消耗,
从而延长终端电池使用时间;并通过抑止无线通信系统中的“远近效应”,
降低网络空中接口部分的相互干扰,提高信道容量。
\end{itemize}

本论文研究所涉及的内容主要集中前三个方面:
包括接纳控制、基站切换和多业务下的资源竞争博弈分析及应用方面。
而且,这些内容也是网络数据链路层的核心技术。
下面我们针对这些与本论文紧密相关的技术研究现状做进一步的阐述和说明。

\subsection{呼叫接纳控制的研究现状}
\esubsection{Development of Call Admission Control}
因为无线资源有限,所以为了降低网络拥塞和呼叫中断概率,我们把限制进入通信网络中的呼叫连接数的机制,称作呼叫接纳控制。
对于一个呼叫连接请求,系统不但要考虑这个连接的服务质量要求、也要考虑现有连接的业务服务质量要求。同时,
还要对整个网络资源运行状况进行评估,最终来决定是否接纳该连接\cite{Ahmed2005}\cite{Ghaderi_Boutaba_2006}。
由于呼叫接纳控制可以有效地减小网络的业务负载,保证现有连接业务的通信质量,所以该功能模块对于无线通信而言十分重要。

一般而言,移动通信系统中,衡量系统性能的呼叫接纳控制指标主要有以下三种:
\begin{enumerate}[(1)]
\item 呼叫阻塞率  (Call Blocking Probability,CBP),指当有新用户的连接请求时, 网络拒绝其请求的概率。
\item 呼叫掉线率  (Call Dropping Probability,CDP),指当现有用户正在通信时,系统由于没有足够的资源来维持其通信过程,而不得不中断其通信的概率。
\item 带宽利用率 (Bandwidth Utilization, BU),指呼叫进入网络后,所有被分配出去的带宽资源之和与系统的总带宽资源之间的比值。
\end{enumerate}

在早期呼叫接纳控制的研究中,研究工作主要集中在信道预留和呼叫请求排队与调度方面。

基于信道预留方法的本质是对系统中不同的用户(如新用户,切换用户或是其它类别的用户)加以区分,在资源管理时对不同类别的用户分别预留一定数量的资源。
这里所指的资源可以是带宽、子信道数目或传输的功率。
它的优点是可以显著地提高切换用户的切换成功概率。
根据预留策略的不同又可细分为传统的信道预留(Conventional Guard Channel, CGC)
\cite{Hong:1986}\cite{Lunayach:1982}\cite{Posner:1985},比例信道预留(Fractional Guard Channel,FGC)\cite{Ramjee:1997}\cite{Y-G-Fang.TVT.2002}\cite{Vazquez:2006},受限比例信道预留(Limited Fractional Guard Channel,LFGC)\cite{CruzPerez:1999}以及统一比例信道预留(Uniform Fractional Guard Channel,UFGC)\cite{Beigy:2004}。
这类方法有一个共同的特点,
就是要对不同用户(新用户或切换用户)的信道占用时间做出较为准确的估计,
然后依据其估值对信道进行静态或动态预留。
最简单的方法是由研究者Hong和Rappaport在1986年提出的\cite{Hong:1986}。
他们假定新的用户与切换的用户占用信道的平均时间是相等的,并基于这个假设为不同类别的用户预留资源。
当然,这种方法的缺点是用户占用信道的平均时间估计其实并不准确。 所以,后来的学者采用了不同的方法来对其进行改善。
学者Zhang和Yavuz等人使用一维或二维马尔可夫链模型来进行估计\cite{Zhang:2003}\cite{Yavuz:2006}\cite{Sindal:2008}。
学者Lee等人进一步提出可以用隐马尔可夫来解决动态预留的问题\cite{LeeWu2006}。
同时,学者Fang和Zhang则采用使用新用户与切换用户的信道占用时间归一化来提高估计的精度 \cite{Y-G-Fang.TVT.2002}。
学者Del和Re认为可以对用户的类别信息增加权重来进一步改善精度\cite{Del1995}。
此外,还有学者使用一个更为复杂的两维状态空间和相位合并算法来估计信道占用的概率分布函数\cite{Melikov:2006}。
这些研究都对接纳控制的发展起了很大的贡献。但是这些方法大多只针对语音业务,对其它业务的情况考虑较少。

所以,
随着通信技术的发展和承载业务种类的逐渐增多,通信系统需要能够处理多种业务的呼叫接纳控制算法。
以往传统的只针对话音业务的呼叫接纳算法已经不能满足实际应用的要求,所以,支持多媒体业务(话音、视频等)的接纳控制算法也引起了研究者的重视。
人们从业务的优先级、分类等方面做了大量的工作。
学者Tseng和Hgiao首先在ATM(Asynchronous Transfer mode)网络中将话音和普通数据流通过优先级区分开。
它把话音数据用马尔可夫模型描述,而把普通数据流通过一个混合的泊松模型描述;并且给话音更高优先级来使用网络资源\cite{Tseng:1991}。
这种简单的方法被学者Peha进行了扩展,它引入了一个优先级令牌库(priority token bank)的思想,
通过动态地调度令牌来更加公平地对待每一种业务\cite{Peha:1993}。
学者Iera和Fernandez等人又从数据的流量特点来提出了突发型的优先级概念\cite{Iera:1996}\cite{Fernandez:1997}。
学者Sen等人提出方案中区分了实时和非实时业务。
他们通过建立一套QoS参数的系统效用函数,确定最佳的方案。
同时,在接纳控制方案中采用这个效用函数,可以最大程度地改善无线资源的利用率\cite{Sen:1998}。
学者Yan和Xu则简单地把呼叫分为窄带用户与宽带用户,将每一类用户的到达视为一个随机变量。
并且,将呼叫阻塞概率等评价指标与之相关联,建立模型来研究\cite{Yan:2008}\cite{Xu:2007}。

除了考虑业务本身以外,研究人员还对信道质量对业务的要求也包括在内。 例如,
有的研究者在其提出的方案中引入了功率控制的思想\cite{LiuZhang:2002}\cite{ZhangFang:2006}。
学者Lee和Ki在CDMA系统中先将多媒体业务中的可变码率业务(Variable Bit
Rate,VBR)要求对应为信道质量要求,提出基于信道质量预测的呼叫接纳控制的方案\cite{Lee:2004}。
学者Wha和Dong把用户的业务类型映射为无线信道的质量要求SIR,从而对呼叫接入进行管理\cite{JeonJeong:2001}\cite{Wha:2002}。

除了上述的方法之外,优先级队列也是处理呼叫接纳控制的一种技术。
其思想是在一个较短的时间内,让新用户和切换用户进行排队,然后再对不同的用户进行处理。
学者Yu和Wu等人提出一种概率优先级队列的思想。
通过考察切换用户和新用户在基站中的排队情况,然后给予切换用户较高的优先级。
这种方法可以对呼叫阻塞概率和呼叫中断概率做出较好的折衷\cite{Yu:2006}。
研究人员Ling在排队的同时也考虑了业务本身的一些特点\cite{Ling:2009}。
但是这种方法可以会使得实时的业务造成较大的延时。

%还有研究人员提出可以将呼叫接纳控制的过程分成两个阶段。在第一阶段考虑业务本身的带宽需求;在第二个阶段考虑业务的延时需求\cite{Myung:2003}。

由于视频等宽带多媒体技术的发展,人们将应用层信源编码的技术引入到呼叫接纳控制算法中来。
当无线信道的质量下降或是基站业务负载过大时,通信系统可以调整信源编码来降低码率。
这样会使得这些业务的速率可以动态调整,系统容量也不再是一个固定的,
而是在一定范围内波动。
例如,学者Xiao和Chen在方案中引入了降级区间的思想(Degraded Area Size)\cite{XiaoChen:2000}。
他们把降级的程度和降级的时间相乘的积做为一个新的控制参数。
学者Kwon通过最小化降级概率的算法来得到一个可行的方案解\cite{KwonChoi:1999}。
另外,异构网络的出现也使得接纳控制技术更加复杂。
比较典型的是学者Li和Chao提出的自适应异构移动网络的接纳方案\cite{LiChao:2007}。
该方案针对异构融合环境,提出了多业务Qos保证的联合呼叫接纳控制算法。
该算法通过理论建模确立网络中各业务用户的数目的概率分布。
然后又推导出切换每种业务用户切换的概率和用户服务的中断概率,最后将此结论应用于呼叫接纳控制方案中。 
但是,由于接入选择本身己经是一个计算复杂度较高的过程,
如果再与概率计算控制过程相结合就会进一步增加算法的计算复杂度。


\subsection{切换控制的研究现状}

\par
众所周知,目前的无线通信网络拓扑结构都几乎毫无例外地采用了蜂窝的形式,在空间地理上实现频谱的复用。
这也使得通信系统要支持一个移动台从一个基站向另一个基站的切换或是从一个通信的信道切换到另外一个信道。
不同的系统切换的内容稍有不同。
例如,对于时分系统(Time Division Multiple Access,TDMA)切换的实质是时隙的切换。
对于频分系统(Frequency Division Multiple Access,FDMA)切换的实质是频率的切换。
而对码分系统(Code Division Multiple Access,CDMA),切换的实质是码字的切换。

对于切换算法研究,大体可分为两个主要的方向:一个方向是对切换信道的参数进行研究,另一个方向是对切换时机的判断方法和切换的协议设计进行研究。

对于切换参数的研究主要集中在基于信号强度的算法方面。
早在1988年,学者Kanai和Furuya提出了基站应该接纳与之接收信号强度最大的移动台\cite{Kanai1988}。
这种算法的优点是简单。移动台总是与信号最好的基站进行通信,通信质量容易得到保证。
但是其缺点也十分明显。
首先,如果无线信号的质量在移动的过程中频繁的波动,
这种简单的方案会使移动台不停地从一个基站切换到另一个基站。
如果移动台在小区边缘附近,会造成切换的次数过多,进而影响通信质量。
其次移动台在小区边界时,在与相邻多个基站的信号强度相当的情况下,它应该尽量与当前的基站保持连接,而不是立即进行切换操作。
为了改善这种简单方案的不足,学者Senarath和Corazza引入了一个延时的机制(Hysteresis Scheme)\cite{345338}。
他们提出应使切换次数与切换延时之间做一个必要的折衷。
这种切换机制的实质是让切换的次数在时间轴上做了概率平均。
接着,学者Corazza对于信号强度的采样与延时之间的关系做了进一步的分析和扩展\cite{345424}。
为了能更准确地估计移动台的信号强度的状态,学者Sampath采用在移动台上基于信号强度的方差的方法来估计阴影衰落,进而对信道质量做评估 \cite{368926}。
基于这个方法,Vijayan等人又对切换算法参数进行了优化\cite{528368}。
除了通常阴影衰落以外,研究人员对于信号强度的瞬间变化也做了更为细致的分类工作。
例如,在街道的拐角处或是突然进入一个大楼内,系统也应该对这种变化做出及时的响应\cite{350270}。
这其中,典型的工作是由学者Dassanayake完成的\cite{220848}。他总结了无线信道在上述情境中的变化特征,并提出相应的切换参数。
而学者Pollini和Grimlund等人将延时的概念更细化到一个宏小区或是一个微小区\cite{486807}\cite{140543}。

除了延时机制外,研究者还对是否启动基站切换的信号强度阈值以及时机做了深入的研究工作。
其核心的思想是在一段时间内对信号强度的测量采样多次,形成对移动台的信道状态做出更为准确的估计,为切换提供更多的信息。
通常而言,如果当信号强度低于-95dBm时,就要进行切换;否则通信就会中断。
研究者提出了两个较为复杂的模型:基于信号强度的路径损失斜坡模型和功率水平往复波动统计模型\cite{William1995}。
通过对斜坡函数的倾斜程度和往复次数的统计来判断移动台的移动情况,进而适时地启动切换过程。
例如,当斜坡函数表现为较为陡峭的曲线,那么切换操作也要立即开始。如果相反,说明移动台移动速度较慢,切换也不用立即开始。
这两种模型在一定程度下也能减少不必要的切换过程。但是同时也有缺陷。
例如,如果基站接收到信号的强度大是由于干扰的原因,那么本该进行的切换会被误判。
还有研究人员提出了双阈值的方法,把切换的判断过程分成两个部分。
当信号强度降低到第一阈值时,切换的请求可以发出,同时评估过程也会随之进行。
如果目标基站的信号质量好,那么就切换;如果不如当前基站,那么切换就暂停。
当信号强度降低到第二个阈值时,此时不再比较当前基站的信号质量和目标基站的信号质量,切换必须立即进行。
根据上述方法,学者Beck提出一个寻找切换最佳阈值的方法,来降低不必要的切换次数\cite{40070}。

\par 除了对信号的强度直接测量外,信干比(Signal to interference ratio,SIR)也可用做切换判断的参数。
比如,在GSM系统中,SIR通常是$12dB$,对于AMPS(Advanced Mobile Phone System )系统, 这个值会达到$18$dB。
如果当前基站的信干比降低到某个阈值以下,而与此同时另外一基站的信干比又满足通信的要求。
这时,切换就会发生。学者Reig在这方面做深入的研究\cite{1658431}\cite{1390624}\cite{1370826}。
他指出,SIR可以更好的描述信道的质量,为切换决策提供更为详细的信息。
但是同时这种方法也是有缺陷的。例如,在小区边缘时,由于当前基站的信干比和目标基站的信干比都会比较差,可能会导致切换不会发生。
为了避免发生这种情况,一般会加大信号的发射功率来增大信干比。这样的做法又会对其它小区或其它用户的通信造成不利的影响。
所以,学者Chuah提出一个既带有功率控制的功能,又基于信干比的切换算法来解决一矛盾\cite{504935}。


\par 还有一类算法,是基于物理上的距离切换算法。移动台总是与离它最近的基站进行切换。
最早,距离的测量是通过比较无线电波传播的延时来间接获得\cite{Rolle1986}。现在又有通过其它的手段如GPS等来得到距离参数\cite{5349100}\cite{4534769}。
这种方法的优势是:如果测距准确,切换也会更加合理。但是缺点也明显,无线传播环境复杂,通过无线电波来测量距离的精度很难得到保证。
如果采用GPS的方案,要求移动台必须配备有GPS的装置才行,应用的前提条件有些苛刻。

此外,研究者对于切换判断算法的研究也有许多有意义的结果。
譬如有些研究者提出了基于动态规划的方法。
它的本质是先建立一个切换的模型,然后对某些切换相关的参数进行估计,
最后通过优化的方法得到相关的解。
例如,学者Manjari和Wayne把切换过程变换成一个投入与产出的优化问题,
而把接收信号的强度变化被看成一个随机过程\cite{513159}。
他们先将产出函数与信号强度、信道衰落、流量分布、功率控制等条件相关联。然后,通过建立基于资源分配的惩罚函数来对切换过程建立数学模型。
最后,利用动态规划来推导出切换的最佳策略。
他们的仿真实验表明,这个方法比通常的仅仅基于信号强度的算法性能要好。
但是,算法的复杂度较高。
学者Kelly则对切换的期望次数和服务中断失效进行研究,建立了一个切换模型,
也是利用动态规划的方法提出了一个折衷的优化方案\cite{618185}。
学者还提出了一个马尔可夫过程(Markov decision process),并引入了延时的机制来对切换过程建模\cite{504996}。
这类方法有个共同的特点,对切换的场景模型依赖性强,同时在求解过程前要求有一些先验的知识来限定模型。
这样就显得不太灵活。
还有些学者提出了基于模式识别的方法。这类方法利用模式识别算法定位切换场景中的有意义的规律。例如,Maturino就使用模式识别方法将同一类的变量定义成特征空间\cite{345157}。
利用分类的算法来测度切换中可能采用的模式距离。他同时又采用相关的聚类算法更精确地选择切换的最佳基站。
除了这些方法外,还有的学者尝试将模糊理论应用在异构的网络基站切换中\cite{5189770}\cite{5672711}。
总的来说,由于这类方法的复杂度较高,实际应用并不多见。
\subsection{博弈论在资源管理方面的研究进展}
对于有限的无线资源的竞争与分配,博弈论作为一种数学工具也被引入到这个研究领域。博弈论本质上是用来理解和分析理性决策者之间的行为。
它首先应用于经济学领域,对于分析市场中各种经济体的行为起来很大的作用。
如今,无线通信领域的博弈论研究已经十分深入。
它可以应用于不种类型的网络,例如,无线传感器网络、认知无线电或Ad-hoc网络中\cite{MachadoTekinay:2008}\cite{WangWu:2010}\cite{Srivastava:2005}。
也可以应用于不同的功能模块中,
例如,分布式功率控制,无线资源分配或动态竞价等方面\cite{AlpcanBasar:2006}\cite{Senqupta:2009}。
这里,我们从OSI分层的角度来对当前的博弈论的研究进展进行总结与分析。

首先从物理层角度来看,系统的性能通常与无线移动台或无线结点接收到的信号质量,比如SINR(Signal-to-interference-and-noise ratio)紧密相关。
无线结点会根据SINR来调整自己的信号状态。
这时,物理层模块会做出相应的选择或决策。
博弈论就可以用来在这一层次对无线资源,诸如功率或信道频谱进行分配。
学者GoodMan认为功率控制本身就是一个多方博弈的例子\cite{GoodmanMandayam:2000}。
他在文章中指出通过增加功率的方法来改善自己的SINR是最简单的方法。
但是这种做法能够有效是有一个前提的:
接受基站服务的用户集合中其他的所有用户都不会调整自己的功率。
因为增加自身的功率同时必然会造成其他用户的信道质量下降。
所以,现实可能会是:所有的用户都想改善自己的信道质量,都会想试图使用增加功率的方法。假设大家都增加了功率,那就会造成所有人的信道都会因他人的功率增加而下降。最后,所有用户都没有从这种方法中受益。
学者Mackenzie和Wicker在CDMA系统中设计了一个完全信息下的单小区功率控制非合作博弈方案\cite{MackenzieWiker:2001}。
在这个方案中,电池中有限的电量被考虑进来。如果用户的传输功率太高,那么电池的电量会被很快耗尽。如果所有人都这么考虑,电池的电量被白白耗尽了,但传输质量却没有提高。所以,用户在权衡自己通信方案的同时,还要考虑其他人可能采取的方案。作者最后通过重复博弈的机制,找到一个大家都认可的纳什均衡解。
学者Gunturi和Paganini将这个方案又应用到多小区的场景\cite{GunturiPaganini:2003}。
%而学者Zhu Han又将此多小区博弈方案中增加了一个裁判角色来做进一步改善工作\cite{HanZhu:2007}。

除了CDMA系统的网络外,OFDMA系统中功率控制也应用到了博弈论。
在这些应用中,功率控制的目的是根据不同用户需求,分配不同的子信道数来尽可能保持总的功率最小。
学者Zhu Han 将这个问题归结为一个基于用户之间的非合作博弈注水算法,并且找到了一个功率控制的最优解\cite{HanZhu:2007}。
同样的,与此类似的方案也被扩展到了多小区的情况\cite{WangXue:2006}。
这个方案通过增加惩罚函数而达到帕累托(Pareto)的改善。
另一个重要的内容是频谱分配。对于多个设备如何有效地使用无线频谱也是一个研究的热点问题。
这个问题常常会被归结为在资源的最大化利用与公平之间的矛盾\cite{JiLiu:2007}。
学者Suris等人提出一个分布式合作博弈的方案\cite{SurisDasilva:2007}\cite{SurisDasilva:2009}。
他们分析了在多跳无线网络中的频谱分配公平问题。
然后,将博弈者的效用空间归结为一个近似的凸空间。
根据这一点,将一个分配问题转化为一个纯策略的博弈问题。
学者Leshem和Zehavi也是采用了类似的思想,将凸规划问题应用到Nash解中
\cite{LeshemZehavi:2008}。 学者Niyato和Hossain则用一个简化的垄断博弈模型来处理认知无线电中频谱分配与公平的问题\cite{NiyatoHossain:2008}。
它的特点是不但将频谱的使用者视为了博弈参加者,而且将频谱的提供者也视为博弈参加者。
最后通过迭代的方式得出一个各方都满意的解。
除此之外,竞价的思想也被引入到博弈中来。
Niyato等研究者又它应用到异构网络中:多个服务提供商的频谱资源竞价问题 \cite{Niyato:2008}。
这里博弈者变成多个服务提供商,而不普通的用户。他们通过建立一个Bertand博弈模型,让各个服务商提供自身相应频谱的价格给普通用户选择。最终的博弈结果可以收敛到纳什均衡。

此外,博弈的思想还被MIMO(Multiple   Input   Multiple   Output)系统以及合作通信技术所使用\cite{LiangDandekar:2007}\cite{ChenKishore:2008}。

其次是博弈论的方法在MAC层中应用。
与物理层类似,博弈论的应用是通过在MAC层上所使用的通信技术来体现的。
例如,在Slotted Aloha网络中,共享资源不再以物理频谱来衡量,而是用户的选择“传输或等待”来衡量。
学者MacKenzie和Wicker基于所有用户的选择来建立自身的效用和收益函数\cite{MackenzieWiker:2001}。
这些收益函数不但取决于用户自身,也与其他的用户相关。

学者Simeone和Barness通过随机博弈理论分析了认知网络中一个2X2的干扰信道。提出数据包到达的随机性与信道占用的随机性对传输性能的影响很大。所有用户都想尽可能获取对信道的使用权来最大化自身的效用。用非合作博弈的方法可以有效地解决信道占用不公平的问题\cite{SimeoneBarNess:2007}。

再次是网络层中对博弈论的应用。
无线网络层中的最主要问题是路由选择或是数据包转发的问题。
路由问题被看作为一个零和博弈。博弈者被认为是路由器与网络自身。
一个博弈者的收益最大时也是其对手的收益最小的时候。
在这种情况下,零和博弈解或均衡点的取得是当博弈参与者的最大最小值(maxmin value)等于它的最小最大值(minmax value)。
包转发的研究主要集中在Ad hoc的网络中。
从理性的角度看,无线结点往往从自身考虑而不愿意为他人转发数据包\cite{Pavlidou2008}。
为了解决这一问题,研究者通过人为设置一些让结点合作的激励机制。
学者Buttha提出可以用“虚拟货币”的方案\cite{Butty_Hubaux_2003}。
如果一个结点请求别的结点的服务,它就要付给别的结点货币。
同时,它当为其他结点提供转发服务时,它会得到相应的收益货币。

学者Buchegger和Michiardi则通过建立和维护一个信用系统(repuation system)来试图解决自私的问题\cite{Buchegger_Boudec_2002}\cite{Michiardi_Molva_2002}。
每个结点都维护一个自己的信用。其它结点对此结点提出转发请求根据自己对该结点的信用了解程度来做出相应的决策。
另外,学者Srinivasan认为博弈过程中可以将各个结点的能量分类,再考虑转发路径上所需要的总能量。并且提出了一个分布式算法GTFT(Generous Tit for Tat)来求得一个转发路径的纳什解\cite{Srinivasan2003}。
Felegyhazi扩展了上述的工作,通过多次迭代博弈来求得一个纳什解\cite{FelegyhaziHubaux2006}。

再次是传输层上对博弈论的应用。
这类博弈论模型主要是研究和分析在传输层上拥塞算法。
通过限制新的会话来控制网络流量并解决拥塞。
根据博弈参与者的角色大体可分为两类。
第一类是异构或同构的网络主体是博弈参与者和竞争者。
对于用户提出的服务请求,不同的网络选择最适合自身的服务请求\cite{Charilas2009}。
每次有一个服务请求被接受。博弈过程不断重复,直到所有请求都被接受为止。
第二类是博弈参与者则包括用户以及网络。
这类方案的目标是不但要使网络的性能得到最大化的利用,同时也要使得每一个用户自身的服务质量尽可能公平。在它们网络模型中,会出现多个用户以及多个网络。因为用户的收益与网络提供者的收益是一对矛盾,所以这类问题通常会看成一个零和博弈问题\cite{LinChatterjee2005}\cite{VlacheasCharilas2008}。

对于应用层而言,学者Park和Schaar的研究集中在多个视频编码的码率分配问题\cite{ParkSchaar:2007}\cite{ParkSchaar:2007ICASSP}。
他们通过提出一个基于率失真优化的博弈模型,通过议价博弈形式使得所有视频流的分配到的码率达到帕累托最优。

\section{论文主要研究的内容与组织结构}
\subsection{论文的主要贡献}
本论文研究的内容主要针对移动多媒体通信网络,
特别是对当前技术发展的几个关键点进行了深入细致地研究。
其中包括高速移动台切换协议流程及信令保护,呼叫接纳控制以及相应的资源调度机制,
以及应用博弈论方法解决多媒体通信网络中资源分配和调度的研究工作。
论文的主要贡献总结为以下几个方面:
\begin{enumerate}[(1.)]
\item 
针对高速移动台切换问题,建立了以移动速度为基础的切换成功概率模型。
通过该模型来考察速度对切换协议信令影响的关系。
提出一个自适应的FEC保护方案来匹配不同移动速度,进而改善切换成功率。
仿真实验证明该方法简单有效。

\item 
根据多媒体业务的不同,
建立一个针对网络底层资源分配参数与网络高层的业务服务质量的映射模型,
实现不同业务用户的QoS测度以及系统整体性能的有效测量。
并且,基于这个映射模型,提出了一个基于系统效用和用户自身效用兼顾呼叫接纳控制与资源分配的算法,
实现不同业务Qos以及系统资源的在线管理。
该算法以用户服务质量效用为最大化目标,在高负荷下依据用户业务负载情况调整接纳与分配策略。

\item 
针对网络中业务逐渐增多的趋势,提出通过概率连续随机变量来描述用户的业务类型。
与传统的离散分类不同,新的方法可以连续细致地刻化用户业务特征。
并且根据网络的实际情况,提出在不完备信息的情况下通过构造Bayesian博弈模型的方法来寻求使用户满意的资源分配策略。
通过理论分析与仿真实验证明,所提出的业务描述方法及博弈算法,可以有效地解决在多用户竞争下的资源分配问题。

\item 

\end{enumerate}

\subsection{本文的组织结构}
本论文的工作集中于MAC层上的无线资源管理与控制。
论文的各章节安排如下:

第一章对当前无线通信发展背景做了简单的介绍,并详细描述了在无线宽带系统发展中的所遇到的难点及热点。
然后分析了资源管理和调度系统目前存在几个关键技术的研究发展现状。
在此基础上,提出本论文的研究内容以及各章节的具体安排。
\par 第二章介绍有关无线网络传输的基础结构和知识,探讨了MAC层的资源管理与控制的特点以及与应用层中多媒体内容相关的知识,为本论文工作的开展提供了必要的概念和知识准备。
\par 第三章以WiMAX网络为例,描述了MAC层中无线终端在切换过程的协议以及在其中传递的信令。
然后研究了对于高速移动的终端的误码率对于切换信令的影响,并建立了相应的数学模型高速移动用户基站切换算法问题。对于用户切换的流程进行分析,建立了一个WiMAX基站切换信令协议的交互概率模型。通过对此模型的分析,提出了一个用于在高速移动速度下保证切换成功的前向纠错方法。该方法在提供了一个方法通过增加额外的保护来保证切换过程的完成。通过仿真实验表明,所提出的方案可以通过计算在不同的移动速度下达到所需冗余比特数来达到认定的基站切换成功率。
\par 第四章研究了多媒体业务用户的资源分配与数据调度问题。
本章首先分析了多媒体业务的出现导致不同业务数据对服务质量的要求均有所不同。
同时,对同一业务类型的不同数据流,用户对服务质量的要求也不同。
由于用户对于服务质量的感受是更多是从应用层,所以我们首先提出了一个针对不同业务及不同数据流的服务质量的折算方法。
通过这个方法将应用层QoS对资源的要求映射为一个归一化的数值。
然后,利用凸规划算法,给出了目标问题的求解方法。并基于此方法提出一个新的呼叫接纳控制算法。
最后,通过仿真对这一资源分配方法及呼叫接纳控制算法的QoS保障情况,系统的资源利用率等性能进行了分析。
\par 第五章研究了在非完整信息下的资源分配博弈问题。
在用户信息不完备的情况下,通过用户博弈的方式研究用户类型对无线资源分配与管理的影响。
我们首先提出一个新的业务类型描述方式。
与以往的离散业务类型描述方法不同,我们将用户的业务类型描述成一个连续的随机变量。
然后建立了一个基于Bayesian博弈的竞争与决策分析模型。
这个模型可以有效地描述在不完备信息下的用户资源竞争的情况。
通过理论分析来研究用户的业务类型对博弈结果的影响。
通过理论分析得知,建立适当的收益机制,可以激励用户根据自身的业务情况做出理性的分析和决策;既可以保证自身的服务质量,
又能让资源得到充分利用。
\par 第六章研究了基于Nash议价的多用户资源分配方案。

\par 第七章是本论文的结论部分。
对全文的工作进行总结,并对与论文进一步的相关研究进行了展望。
