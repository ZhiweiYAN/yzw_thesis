\chapter{基于议价博弈的多用户资源分配}
\label{chap_nash_bargain}
在带宽受限的网络中,多用户的多媒体应用是十分常见的。
诸如视频点播的数据流、在线游戏或网页浏览等等的应用。
为了能够保证业务的正常进行,资源分配的算法要保证应用的各种服务质量要求,比如时延、带宽等。
本章我们通过建立基于议价合作的博弈模型,
公平地把系统资源分配给网络中的各个用户。
对于我们的问题,Nash议价解是一个公平且优化的解决方案。
这个解可以使系统的效用达到最大化,同时又能保证各个用户资源分配的公平性。

\section{前言}
议价博弈理论是微观经济分析理论的一个分枝。
它的最初提出原因是为了要解决双边垄断市场结构下价格的决定问题。
例如,对于员工的工资和就业的决定问题。
其中,工会是代表员工利益的代理人,同时也是垄断的劳动供给者。
经理人是企业的代理人,也是垄断的劳动需求者。
在传统的经济学供给需求分析框架中,我们不能确切地知道最终的均衡工资和就业程度到底在什么位置。
传统的理论只是模糊地说,均衡的工资和就业程度取决于企业经理人和工会的谈判能力,也就是所谓的“议价能力”。
但是议价能力的具体描述则取决于一些比较模糊的因素,例如工会组织的参加率、双方的谈判次数、市场对谈判的期望等等。
议价理论的出现,使得经济学家们可以用数理化的方式解决这些类似的问题。

议价理论在博弈论中十分重要,它更加完善地考虑到了个人理性与集体理性的问题。
itemize
这里的个人理性是指:对于博弈双方而言,
在双方在议价解下的所得的效用要必须高于双方不进行博弈的所得效用。
也就是说博弈者为了得到更高的收益主动地进行议价博弈。
集体理性是指:议价最终结果应该是具有帕累托最优性质。
任何一方都不可能在不损害对方利益的情况下增加自己的收益。

\section{议价博弈模型}


