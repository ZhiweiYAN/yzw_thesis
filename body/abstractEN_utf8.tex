\eabstract{
With the development of the Internet in recent years, 
new Internet applications such as Facebook, Twitter, attract people attention all directions in their lives.
People are experiencing the changes of their lives pattens introduced by new technologies of broadband network and wireless networks.
Popular demand is to keep connecting them to the Internet anytime and anywhere. 
However, the scarcity of radio-spectrum resource does challenge the demand of the increasing multiple media traffic. 
Therefore, it also challenges the intelligence of researchers for radio resource management.

This dissertation focuses on the resource management in data link layer over wireless networks, especially for the following  important issues: 
handover algorithm with high velocity mobile station,  the mechanism of resource allocation in call admission control, and the application of game theory on resource allocation.
The main contributions of this dissertation include the several parts in the following:
\begin{enumerate}[(1)]

    \item According to the characteristic of multimedia traffic, the mapping model is proposed to examine the relationship between the resource management configuration in the data link layer and the quality of service in the application layer. And we develop a new call admission control algorithm with the proposed model. The algorithm manages both ongoing connections/calls and new ones efficiently. It also balances the utility of individual connections and the utility of a system, especially under heavy load traffic.  

    \item In order to deal with the competition of resource among selfish users over wireless networks, we proposed a new competing model of game theory and its resource allocation algorithm. The game model introduces the Nash bargaining solution to solve the problem of competition. The proposed definition of utility of users can construct the convex utility set, which satisfies the Nash axioms. Then the bargaining powers of users are discussed. The results of simulations confirm that the game model is valid, and that the resource allocation among selfish users are fair. 

    \item We introduce a new method that the characteristic of traffic of user can be described by a continuous random variable. The method can  handle the fact that the number of traffic type is increasing sharply. Thus, the range of description is transformed from discrete form to continuous one . Then, we analyze the change and  build a Bayesian game model with incomplete information to handle the change.
Through adaptively adjusting the estimated traffic type random distribution, we can allocate the resource into users more efficiently than before.

    \item The probability model of handover between base stations is built with the velocity of mobile station. We investigate the relationship between the velocity of mobile station and the handover operation flow by means of the model. The probability of successful handover is mainly related with the bit error rate and the distance of overlap between base stations. Then, a new adaptive  redundancy protection scheme with forward error correction is proposed to improve the probability of handover success according to the velocity of mobile station.
\end{enumerate}

}

