\eabstract{
With the development of the Internet in recent years, 
popular demand is to keep their links to the Internet anytime and anywhere. The new Internet applications such as Facebook, Twitter, attract the people attention in their normal life.
People are experiencing the changes of their lives influenced by new technologies about broadband network and wireless networks.
However, the scarcity of radio-spectrum resource does challenge the demand of the increasing multiple media traffic from users. 
Therefore, it also challenges the intelligence of researchers for radio resource management.

This dissertation focuses on the resource management in data link layer over wireless networks, especially for the three important problems: 
handover algorithm with high velocity mobile station,  the mechanism of resource allocation in call admission control, and the application of game theory for the resource allocation.
The main work in this dissertation includes the several parts in the following:
\begin{enumerate}[(1)]

    \item According to the characteristic of multimedia traffic, the mapping model is proposed to examine the relationship between the resource management configuration in the data link layer and the quality of service in the application layer, perceived by  users. And we develop a new call admission control algorithm with the proposed mapping model. The new algorithm can manage both ongoing connections or calls and new ones efficiently. It also balances the utility of individual users and the utility of a system, especially under heavy load traffic.  

    \item In order to deal with the competition of resource among selfish users over wireless networks, we proposed a new competing model of game theory and its resource allocation algorithm. The game model introduces the Nash bargaining solution to solve the problem of competition. The proposed definition of utility of users can construct the convex utility set, which satisfy the Nash axioms. Then the bargaining powers of users are discussed. The results of following simulations confirm the validity of the game model, and the allocation of resource among selfish users are fair. 

    \item We proposed a new concept that the characteristic of traffic of user can be described by a continuous random variable because the number of traffic type is increasing sharply. It changes the range of description from current discrete traffic descriptions. Then, we analyze the change and  build a Bayesian game model with incomplete information to depict the situation.
Through adaptively adjusting the estimated traffic type random distribution, the resource management unit can allocate the resource into users more efficiently than before.

    \item The probability model of handover between base stations is built with the velocity of mobile station. We investigate the relationship between the velocity value and the handover operation flow by means of the model. The probability of successful handover is related with the bit error rate and the distance of overlap between base stations. Then, a new adaptive  redundancy protection scheme with forward error correction is proposed to improve the probability of handover success according to the velocity of mobile station.
\end{enumerate}

}

