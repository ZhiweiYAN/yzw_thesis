\par 本学位论文是在刘贵忠教授的亲切关怀和悉心指导下完成的,谨向刘老师表示衷心的感谢,并致以崇高的敬意!
刘贵忠教授渊博的理论知识、严谨的治学作风、求实的研究态度、
谦逊的学者风范、孜孜不倦的求是精神以及在学术领域的远见卓识都令我深深敬佩。
在我的博士阶段学习和科研工作期间,刘贵忠教授给予了无微不至的指导、关爱和鼓励;
当我遇到困难时,他宽厚待人的态度、对学生的关怀与鼓励,使我在学习期间保持斗
志和信心。

在人生漫长的求学生涯期间,能在刘贵忠教授的教导下学习和工作,我感到十分的荣幸。
我所取得的每一点进步,倾注着导师的心血。
刘老师高尚的人格,渊博的学识,一丝不苟、富有启发性的治学作风,坚韧不拔、锐意进取的工作精神,民主而严谨的治学作风将是我永远学习的楷模!

\par 在过去几年中,十分幸运地能和~$SIGPRO$~实验室及~$3C$~实验室中许多优秀的同学一起学习交流。
在此,首先感谢李瑛、武林俊、苏睿、唐耀华、田小永、陈志刚、孙晓东。
其次要感谢留在交大继续工作的钱学明、赵凡、李锋、李凡。
还要感谢曾经与我朝夕相处的实验室的同学,张庆、张静、王喆、南楠、孙力、陈立水、韩一娜、谢辉、程逸逸、戈晓旦、王凤玲、高毅欣、姜海侠、张娜、郭旦萍、刘占伟、李智 、 王海东、 张益民、 王琛、党红强、任斐斐、 汪欢、 金剑、 胡瑛、 贺丽君、马亚娜、 张海涛 、王星 、杨阳 、廖开阳 、王秦立 、蔡秀霞 、邱明建 、肖丽 、惠有师。他们一直全力支持我的研究和学习,也给我的业余生活带来了许多快乐。

\par 同时,感谢南加利福尼亚大学(USC)的Prof. C.-C Jay Kuo。
尽管与Prof. Kuo的交往只有一年的时间,但是~Prof. Kuo~严谨的学风、认真扎实的科研态度使我受益匪浅。
感谢Loyola Marymount University的Prof. Lei Huang,Huang老师对我的研究工作给予耐心细致的指导和不断的鼓励,每周不但要花大量时间阅读我的周报,还对我在USC的生活给予诸多的方便。在此对黄老师为我付诸的心血和期望表示由衷的感谢。
同时,也要感谢华中科技大学的徐士麟、北京大学的刘家瑛,一起在南加州大学的学习与生活令人难以忘怀。

\par 衷心感谢我的各位师长、学友,以及我的朋友们对我的关心和帮助。 
\par 要特别地感谢在故乡的父母和兄长!在我漫长的求学生涯中,他们始终在物质上尽一切可能给予支持,在精神上给予不断的鼓励与鞭策!

\par 最后,还要感谢我的妻子和孩子。他们的爱和理解才使这一切成为可能。
