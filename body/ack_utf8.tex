\par 本学位论文是在刘贵忠教授的亲切关怀和悉心指导下完成的。 首先谨向刘老师表示衷心的感谢,并致以崇高的敬意!
刘老师渊博的理论知识、严谨的治学作风、求实的研究态度、
谦逊的学者风范、孜孜不倦的求是精神以及在学术领域的远见卓识都令我深深敬佩。
在我的博士学习和工作期间,刘老师不但在学术上给予了细致的指导; 同时在生活中也给予了许多的关照和勉励。
刘老师高尚的人格,渊博的学识,一丝不苟、富有启发性的治学作风,
坚韧不拔、锐意进取的工作精神,民主而严谨的治学作风将是我永远学习的楷模。

\par 在交大学习期间,十分幸运地能和~$SIGPRO$~实验室及~$3C$~实验室中许多优秀的同学一起学习交流。
在此,先要感谢李瑛师兄、苏睿师兄。两位师兄不但在工作中支持我的学习,同时也对我在交大的生活给予了许多方便。还要感谢侯兴松师兄、赵凡、李小和、钱学明、顿玉洁、李凡、武林俊、唐耀华、田小永、陈志刚、孙晓东、李锋。
还要感谢曾经与我朝夕相处的实验室的同学,张庆、张静、王喆、南楠、孙力、胡瑛、陈立水、韩一娜、
李智、王海东、张益民、贺丽君、王秦立、邱明建等。
%肖丽、惠有师、张海涛、
%谢辉、程逸逸、戈晓旦、高毅欣、姜海侠、 王琛、汪欢、金剑、王星、杨阳、廖开阳、蔡秀霞、邱明建、肖丽、惠有师。
他们一直全力支持我的研究和学习,也给我的业余生活带来了许多快乐。

\par 同时,感谢南加州大学(USC)的Prof. C.-C Jay Kuo。
尽管与Prof. Kuo的交往只有一年的时间,但是~Prof. Kuo~严谨的学风、认真扎实的科研态度使我受益匪浅。
感谢Loyola Marymount University的Prof. Lei Huang,Huang老师对我的研究工作给予耐心细致的指导和不断的鼓励,
每周不但要花大量时间阅读我的周报,
还对我在USC的生活给予诸多的照顾。
在此对Huang老师为我付诸的心血和期望表示由衷的感谢。
同时,也要感谢华中科技大学的徐士麟、北京大学的刘家瑛,一起在南加州大学的学习与生活令人难以忘怀。

\par 衷心感谢我的各位师长、学友,以及我的朋友们对我的关心和帮助。 
\par 同时,要感谢在故乡的父母和兄长。 在我漫长的求学生涯中,他们始终在物质上尽一切可能给予支持,在精神上给予不断的鼓励与鞭策。 最后,还要感谢我的妻子杨琳琳和孩子燕阳天。她们的爱和理解才使这一切成为可能。
