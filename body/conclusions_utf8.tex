%chapter_begin
\chapter{结论与展望}
\echapter{Conclusions and Future Work}
\label{chap_conclusions}
\section{论文总结}
\esection{Conclusions}
互联网技术的发展为多媒体业务的开展提供了广阔的应用平台。
各种新型的互联网应用,也改变了人们对传统互联网通信的认识。这些应用也使得人们希望能始终保持与互联网连接,随时随地得进行自由的信息传递。但是,由于空中接口频谱资源的稀缺,使得无线互联网技术遇到了许多技术难题。

本论文以无线资源管理为主线,以解决数据链路层中所涉及移动性管理和资源的合理公平利用为目标,
针对高速无线移动终端的基站切换问题、基于业务流量特征的呼叫接纳控制算法与资源分配问题,以及博弈论在资源管理方面的应用做了一些有意义的探索。
主要的内容简要总结为如下四点:

\begin{enumerate}[(1.)]
\item
针对多媒体业务用户的呼叫接纳控制与资源管理问题,
我们首先分析了多媒体业务的特点,
建立了一个网络高层服务质量评估与数据链路层质量评估关系的映射模型。
该模型可以针对不同的业务数据流在数据链路层上映射为一个归一化的QoS水平评估指标。
基于这样一个QoS的映射模型,可以有效地评测各个用户连接的QoS水平与资源的需求,并最终提供给呼叫接纳控制单元做为接入控制的判断依据。
而后,所提出的接纳控制算法充分利用了这一模型特点,有效地改善接纳控制的性能。
仿真结果表明,此算法在保证用户服务质量的同时,改善了系统的连接容量、资源利用率、以及新用户的阻塞率。
\item 针对多用户资源竞争的问题,将非对称纳什议价博弈理论引入资源分配问题的解决方案中来。
构造了一个新的资源分配议价博弈模型,定义了博弈参与者的效用,用户的分歧点等。
通过对模型的理论分析,证明了该分配模型满足纳什议价公理所提出的各项约束。
然后,分析了用户议价能力对分配结果的影响,
进而提出以用户应用特征参数值为基础的议价能力的具体定义。
仿真结果表明,所提出的博弈模型与相应的资源分配算法,可以有效地且公平地解决用户资源竞争的问题。所提出的资源分配算法在保证系统资源利用率的同时,也兼顾了各个不同类别用户之间的需求。

\item 研究了在非完备信息下的资源分配Bayesian博弈问题。
针对信息不完整,我们提出一个新的业务类型概率描述方法。此方法不再使用传统的离散的业务描述方式,而是使用连续的概率随机变量的描述方式。。
基于这种描述方式的转变,我们提出并构造了一个基于Bayesian博弈的资源竞争与决策分析模型。
这个模型可以有效地描述在不完备信息下的用户资源竞争的情况。
通过理论分析来研究用户的业务类型对博弈结果的影响。
理论分析结果表明,建立适当的收益机制,可以激励用户在信息不完备的情况下,仍旧可以根据自身的业务情况做出理性的分析和决策。
最后的仿真结果也显示,所提出的博弈模型在平衡用户的资源竞争方面是有效的。

\item 针对WiMAX网络中无线终端的移动速度对切换过程的影响问题,通过分析切换过程的协议以及其中传递的信令流程,
建立了一个基站切换信令交换流程与移动终端的移动速度之间关系的概率模型。
通过对此模型的分析,提出了一个用于在高速移动速度下保证切换成功的速度自适应的前向纠错方案。
该方案提供通过增加额外的保护信息来确保切换过程的顺利完成。
仿真实验结果表明,通过预先计算好不同移动速度下所需的冗余比特数,就可以在不同速度下达到设定的基站切换成功率。
\end{enumerate}

\section{研究展望}
\esection{Future Work}
本文针对无线宽带网络,从系统设计的角度出发,重点研究
了数据链路层中有关资源管理的几个问题,取得了一些有意义的结果。
但是本文所涉及的研究问题都集中在无线网络的数据链路层上。这些问题只是无线资源管理课题中的一小部分,仍然有大量的内容作者尚未涉及。基于目前作者对无线管理领域的了解与认识,认为以下三点值得进一步深入开展研究工作。
\begin{enumerate}[(1)]
\item 业务模型的分类定义与选择问题。解决资源管理与控制问题的先决条件是对用户业务流量特点的理解。由于多媒体业务类型多样且越来越复杂,如何自适应地建立业务分类的概率模型,并得到确切的模型参数。这些内容都值得以后的研究者予以重视。确定用户业务的连续分类方法也许是无线资源有效且公平分配问题中的一个有效的突破口。
\item 无线资源管理在IP架构下的交叉层技术及协议规划问题。目前几乎所有的网络,都向着全IP的网络架构发展。无线资源管理在网络上层(如IP层或传输层)也许可以获得更大的自由度。特别是对于异构网络的大量出现,在IP层可以增加更多的资源管理的内容。这要求在制定新网络基础协议时提供支持,同时又要兼容现有协议。这对于研究者来说是一个挑战和机遇。
\item 非完备信息下的合作博弈在资源分配中的应用。由于信息的不完整与不对称,这使得资源管理与公平分配更加困难。博弈理论工具的引入,可以对资源争用问题更有效地建立数学模型。通过这些模型的研究,可以让研究者摆脱直觉上的资源争用的诸多不确定性,使研究更接近与真实的情况;进而才可以提出更加合理与公平的分配方案。
\end{enumerate}

%chapter_end
