%chapter_begin
\chapter{无线网络与MAC层的资源管理简介}
\label{chap_brief}
在无线通信系统中,无线资源管理(RRM)的主要功能就是保证资源的有效利用,为系统保障用户服务质量(QoS)要求提供有效机制。然而,由于无线通信环境的恶劣性和网络状态的动态变化,网络性能通常呈现随机特性,如何在具有随机性的无线网络中为用户提供满意的服务,是RRM的一个重要挑战。近年来,随着无线通信需求的爆炸性增长,无线资源变得日益紧缺,RRM问题也变得益发复杂和重要。
无线网络与有线网络最大的不同是其有限的频谱资源,而无线用户的数量确一
直在高速持续的增加。如何利用有限的无线资源来满足日益增长的需求,是无线通
信急需解决的一个重要的问题。而无线资源管理和调度则是提高无线资源利用率的
核心,无线资源管理就是对移动通信系统的空中资源的规划和调度。优秀的调度算
法,可以使系统获得更高的速率和性能,更好的Qos体验。如果没有好的无线资源
管理技术,再好的系统也无法发挥它的优势。

\section{无线网络传输模型}
 计算机网络的体系结构就是指计算机网络的各层及其协议的集合,或计算机网络及其部件所应完成的功能。计算机网络的体系结构存在的目的就是使不同计算机厂家的计算机能够相互通信,以便在更大的范围内建立计算机网络。
  国际标准化组织ISO于1983年正式提出了一个七层参考模型,叫做开放式系统互联模型(通称ISO/OSI)。[1]OSI参考模型将整个网络通信的功能划分为7个层次,由底层到高层分别是物理层、链路层、网络层、传输层、会话层、表示层和应用层。每层完成一定的功能,都直接为其上层提供服务,并且所有层次都互相支持。第4层到第7层主要负责互操作性,而1~3层则用于创造两个网络设备间的物理连接。
        一、第1层:物理层
  物理层是OSI参考模型的最低层,且与物理传输介质相关联,该层是实现其他层和通信介质之间的接口。物理层协议是各种网络设备进行互联时必须遵守的低层协议。 
  物理层为传送二进制比特流数据而激话、维持、释放物理连接提供机械的、电气特征、功能的、规程性的特性。这种物理连接可以通过中继系统,每次都在物理层内进行二进制比特流数据的编码传输。这种物理连接允许进行今双工或半双工的二进制比特流传输的通   
  物理层相应设备包括网络传输介质(如同轴电缆、双绞线、光缆、无线电、红外等)和连接器等,以及保证物理通信的相关设备,如中继器、共享式HUB、信号中继、放大设备等。
        二、第2层:数据链路层
  数据链路层是OSI参考模型的第2层,介于物理层与网络层之间,其存在形式分为物理链路与逻辑链路。
  设立数据链路层的主要目的是利用在物理层所建立的原始的、有差错的物理连接线路变为对网络层无差错的数据链路,因此数据链路层必须有链路管理、帧传输、流量控制、差错控制等功能。数据链路层所关心的主要是物理地址、网络拓扑结构、线路选择与规划等。
  数据链路层的数据传输是以帧为单位。在OSI中,帧被称为数据链路协议数据单元,它把从物理层来的原始数据打包成帧。数据链路层负责帧在计算机之间的无差错信息传递。
  数据链路层设备主要包括:网络接口卡(NIC)及其驱动程序、网桥、二层交换机等。
        三、第3层:网络层
  网络层是OSI参考模型中最复杂、最重要的一层。这一层定义网络操作系统通信用的协议,为信息确定地址,把逻辑地址和名字翻译成物理的地址。它也确定从信源机(源节点)沿着网络到信宿机(目的节点)的路由选择,并处理交通问题,例如交换、路由和对数据包阻塞的控制。
  网络层的主要提供以下功能
  1.  路径选择与中继。路径选择是指在通信子网中,为源节点和中间节点选择后继节点,以便将报文分组传送到目的节点。“最短时间”是选择路径的标准。[2]
  2.  流量控制。网络中链路层、网络层、传输层等都存在流量控制问题,其控制方法大体相一致。其目的是防止通信量过大造成通信于网性能下降。
  3.  拥塞控制。当到达通信子网中某一部分的分组数高于一定的水平,使得该部分网络来不及处理这些分组时,就会使这部分以至整个网络的性能下降。拥塞控制的主要任务是保证网络高性能运转,保证子网不被它的用户发送的数据所淹没。
  工作在网络层的设备主要有路由器和三层交换机。路由器通过转发数据包来实现网络互连, 其支持的协议有TCP/IP、IPX/SPX、AppleTalk等。三层交换机使用了三层交换技术,解决了局域网中网段划分之后,网段中子网必须依赖路由器进行管理的局面,解决了传统路由器低速、复杂所造成的网络瓶颈问题。
        四、第4层:传输层
  传输层是OSI参考模型的第4层中,是比较特殊的一层。该层的为源主机与目的主机进程之间提供可靠的,透明的数据传输,并给端到端数据通信提供最佳性能。
  传输层从会话层接收数据,负责错误的确认和恢复,以确保信息的可靠传递。如果有必要,它也对信息重新打包,把过长信息分成小包发送,确保到达对方的各段信息正确无误,而在接收端,把这些小包重构成初始的信息。
  传输层目的在于它既可以划分在OSI参考模型高层,又可以划分在低层。如果从面向通信和面向信息处理角度进行分类,传输层一般划在低层:如果从用户功能与网络功能角度进行分类,传输层又被划在高层。这种差异正好反映出传输层在OSI参考模型中的特殊地位和作用。
  传输层所支持的协议有:TCP/IP的传输控制协议TCP、Novell的顺序包交换SPX以及Microsoft NetBIOS/NetBEUI等。
  五、第5层:会话层
  会话层对高层通信进行控制,允许在不同机器上的应用之间建立、使用和结束会话,对进行会话的两台机器间建立对话控制,管理会话如管理哪边发送,何时发送,占用多长时间等。
  会话层负责协调两个应用进程进行的通信,以便使应用进程专注于信息交互。从OSI参考模型看,会话层之上各层是面向应用的,会话层之下各层是面向网络通信的。
  会话层提供的功能有:为会话实体间建立连接,并组织,同步数据传输。最后通过“有序释放”,“废弃”,“有限量透明用户数据传送”等功能单元来释放会话连接的。[3]
  会话层与传输层有明显的区别。传输层负(下转第237页)
(上接第245页)责建立和维护端到端之间的逻辑连接。目的是提供一个可靠的传输服务。但是由于传输层所使用的通信子网类型很多,并且网络通信质量差异很大,这就造成传输协议的复杂性。会话法在发出一个会话协议数据单元时,传输层可以保证将它正确地传送到对等的会话实体,从这点看会话协议得到了简化。
  六、第6层:表示层
  表示层包含了处理网络应用程序数据格式的协议。它从应用层获得数据,并把它们格式化以供网络通信使用。该层将应用程序数据排序成一个有含义的格式并提供给会话层。这一层也通过提供诸如数据加密的服务来负责安全问题,并压缩数据以使得网络层需要传送的数据尽可能少。
  表示层位于OSI参考模型的第6层,在应用层的下面,会话层的上面。它将数据在计算机内部的表示法与网络的表示法之间进行转换,保证所传输的数据经传送后其意义不改变,因此如何描述数据结构并使之与机器无关是表示层要解决的问题。在计算机网络中,互相通信的应用进程需要传输的是信息的语义,它对通信过程中信息的传送语法并不关心。表示层的主要功能是通过一些编码规则定义在通信中传送这些信息所需要的传送语法。
  表示层负责决定在主机间交换数据的格式,包括:数据加密、数据压缩传输、字符集转换等。在不同的时间,可以使用不同的传送语法,如使用加密算法、数据压缩算法等。
  七、第7层:应用层
  应用层是最终用户应用程序访问网络服务的地方,它负责识别并证实通信双方的可用性,进行数据传输完整性控制,使网络应用程序(如电子邮件、P2P文件共享、多用户网络游戏、网络浏览、目录查询等)能够协同工作。  [4]
  应用层是OSI参考模型的最高层,它为用户的应用进程访问OSI环境提供服务。应用层关心的主要是进程之间的通信行为,因而对应用进程所进行的抽象只保留了应用产程与应用进程间交互行为的有关部分。这种现象实际上是对应用进程某种程度上的简化。
  应用层所承处的网络安全功能可粗分为保密、鉴别、反拒认、完整性等。保密足指保护信息不被未授权者访问。鉴别是指在交换信息之前先要确认对方的身份。反拒认功能主要与电子签名有关,比如对拒绝承认所签约的客户必须惟一的确定电子反拒认,以满足法律手续。完整体是指如何确认
白己所收到的信息是原始发来的信息,而不是被窜改或伪造的。
\section{无线资源管理概述}
分组调度算法可以在满足用户间公平性的前提下,有效提高业务的服务质量
(QoS),进行资源的优化分配用。无线资源管理就是对无线通信系统的空口资源的规
划和调度,目的就是在有限的空口资源下,尽可能的为无线用户提供业务质量的保
证,在各个用户无线信道和业务随随机分布、信道特性不确定的情况下,灵活的分
配和调整当前可用的资源,最大限度地提高无线频谱的利用率。通过对人、工作、
任务、资源等进行适当的安排和规划来达到一定的目地都可以称之为广义上的调度。
而我们所称谓的资源调度,则是指对分配规则和服务次序的一种确定的分配方式,
特别是当某一有限资源被多个对象同时抢占是而发生冲突的时候。
调度过程的核心就是调度算法,无线网络的调度算法从无线网络产生至今也经
历了不断研究和改进的过程。
1997年,Bharghava和Lu等人提出了一种适用于无线网络环境的IWFQ理想公
平调度算法  (IdealizedwirelessFai:Queuing),他们的主要思想是参照了WFQ带有
权重的公平队列算法  (weightedFairQueuing)的设计思路,进行了相关适应无线网
络的改进。针对于无线发生链路的突发性干扰造成的数据流错误,理想公平调度算
法会在无线链路恢复正常传输时做相应的权重方面的补偿,从而达到维护用户间公
平性的目的[,7]。
后来,stinica和zhang等人发现IWFQ的补偿算法并不是十分科学,他们参照
sTFQ算法,提出了一种更适应无线环境的调度算法那:cIF一Q自适应信道的独立分
组公平队列算法(Channel一    eonditionhidependentpacketFairQueuing)。CIF一Q算法
的优点在于其的补偿模型弥补了理想公平调度算法算法中业务流之间隔离性问题,
因此其比理想公平调度算法更加合理。
Ramanathan等人提出的   sBFA(serverBasedFairApproaeh)调度算法通过系统
预留一部分信道带宽来提供了对差错流的补偿。SBFA算法可以保持不同业务流之间
的长时公平性,但缺乏对短时公平性的保障,而且其预留信道的机制也会造成无线
在于的浪费和系统性能的下降。
以上这些适用于无线网络的调度算法的提出打开了无线调度算法的研究思路和
格局。哲学算法虽然都只是单一的考虑了无线链路突发错误因素,但它们为后续研
究打下了一定基础,也指出了研究的方向。随着针对CDMA系统下的分组公平调度
算法和下一代宽带无线通信系统设计的比例公平调度算法的相继提出,无线资源调
度所实现的功能以及相关的研究领域也越来越深入。
未来的移动通信系统需要满足多业务混合传输的需求,要求Qos的区分和保证
机制必须在将来的无线资源调度系统中被支持,这也是无线资源调度算法在未来的
一个主要研究和改进的方向。
2.1.1分组调度算法的评价标准
无线资源管理和调度算法要解决的基本问题就是当不同的用户有多个不同的分
组业务流需要传输的时候,必须找到一个合理的规则来安排每个用户每个业务流的
传输顺序和先后次序,以尽可能的满足每个用户不同业务流的QoS需求。可以看到,
在无线环境下,因为无线系统以及系统资源方面的局限性,我们很难同时满足系统
的总吞吐率、用户间的公平性和分组的延迟等评价指标,这些指标有些时候甚至是
相互对立的,比如要求过短的时延肯定会造成系统缓冲的溢出,影响系统的吞吐量
和Qos等指标。我们在研究无线资源的调度算法的过程中,延迟、公平性、以及吞
吐量是需要主要考虑的几个指标。不同分组业务不同用户占用无线资源的统计结论
可以认为是公平性指标的结果,而吞吐率则是统计系统中用户总的吞吐率。我们采
用一定的调度的规则和算法来保证公平性和吞吐量【’”]。
1、时延和吞吐量
在无线网络中吞吐量和时延是两个必不可少的评价标准,它也是有线网络的主
要评价指标。很明显,获得尽可能小的延迟和尽可能大的吞吐量就是改进无线资源
调度算法的主要目的之一。我们汉子道在固定的窗口长度条件下,如果能够有更低
的时延,那么TCP业务的吞吐量就会更高。但是在无线资源调度中面临的情况却完
全不一样了,尽可能多的分组进入无线传输系统才能增加系统的总吞吐量,这样会
使得所有的无线资源利用率接近于完全饱和。然而这样就会产生出新的挑战:随着
系统的吞吐量越来越高,也会有越来越多的分组数,那么也会随之增加每个传输节
点的等待队列的长度,而这种情况也就意味着分组在队列中的排队时间的大大增加,
即系统的总时延也在增加,所以其在无线网络中有时候是一个矛盾的指标。未来能
够更好的评价无线系统调度算法和资源管理的性能和效率,我们提出了一个利用吞
吐量和延迟来衡量其性能的公式,这个公式如下所示:
这个公式如下所示:
POWCf=
Throuthgput“
0<a<l
Delay (2.1)
这个公式的是基于马尔科夫的排队模型建立的,即分组到达和服务的时间分布
都是服从Possion分布,单一的队列和单一服务窗口的系统,并且我们也假设队列长
度是无限长。在多业务的环境下,也就是多服务窗口的模型中,单队列网络能力公
式必须进行相应的扩展,其形式如下:
PO、VCT
ThroughtPuta0<口<l
d(兄)
(艺界)a
工丫又互
兄~
 (2.2)
其中,d表示系统中队列的平均延迟,兄代表系统总的到达率,i代表服务类的
集合。系统总的平均时延可以表示为如下公式:
歹(“)一音艺、可 (2.3)
可以看出,要提高无线宽带系统中资源分配算法的效率和性能,就需要获得更
大的powe:值。
2、服务质量和公平性
在无线通信系统中进行资源分配时,如果系统中的每个用户都有同等的机会去
获得资源,比如轮询算法的思想,那么用户间的公平性肯定会得到很好的保障,但
是这样却不能保障系统的吞吐量。我们如果采用最大载干比算法来进行调度,则用
户间的公平性得不到保障;如果我们希望用户们能够得到相等数据量的资源,则更
多的服务机会究要分配给较差的信道状态的用户【’9]。显然,两种方法都存在不公平
的问题,那么我们就需要一种标准来衡量用户间资源分配的公平性。我们采用
JainFaimesS指数作为衡量无线资源调度算法公平性的主要指标,其表示如下:
 (2.4)
其中,戈表示第i个用户分配到的资源总数,n表示系统的用户总数。如果毛取
值越高,就可以认为系统有越公平资源的分配结果,其公平性越好

\section{MAC层中资源管理问题与挑战}

%chapter_end
